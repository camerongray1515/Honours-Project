\documentclass[10pt]{article}

\begin{document}
	\section{Evaluation of Existing IT Infastructure Monitoring Systems}
		\paragraph{•}
		This section will review current IT infrustructure monitoring systems and evaluate them on
		several points as follows:
		\begin{itemize}
			\item Support for timeseries monitoring and real time alerting
			\item How they can be configured to monitor custom metrics
			\item How are alert thresholds defined
			\item How configuration and custom code is delivered to nodes (if requried)
			\item How the user configures the system
			\item How dependencies are handled
		\end{itemize}
	
	\subsection{Nagios}
	\paragraph{Timeseries monitoring and real time alerting}
	Nagios is primarily focused at real time alerting and therefore has very little in the way of 
	timeseries monitoring.  Additional plugins are available which can be used to graph metrics
	over time but these cannot be used to make decisions on the status of a given system or service.
	All that is supported in terms of alerting on historical data is to refrain from alerting until
	a given condition has been observed in the previous $n$ checks, there is no support for alerting
	based on trends in historical data.
	
	\paragraph{Support for custom metrics}
	Nagios has support for custom metrics through the NRPE (Nagios Remote Plugin Executor) plugin.
	These plugins can be any sort of executable which prints out a message to represent the data read
	as well as a specific exit code which defines the level of the alert, for example "OK", "Critical"
	.etc
	
	\paragraph{Alert threshold definition}
	Thresholds for NRPE agents must be set on the remote server itself.  These thresholds are passed
	into the remote plugin as an argument when it is executed and are used internally by the script to
	output the approprate alert level.
	
	\paragraph{Code/Config delivery to nodes}
	Nagios does not have any in built functionality to distribute configuration files or plugin code to
	remote nodes. In order to automate this, additional software such as Puppet would be required.
\end{document}