\documentclass[10pt]{article}
\title{This is a title}
\author{Cameron Gray}

\begin{document}
	\maketitle

	\section{Evaluation of Existing IT Infastructure Monitoring Systems}
		\paragraph{•}
		This section will review current IT infrustructure monitoring systems and evaluate them on
		several points as follows:
		\begin{itemize}
			\item Support for timeseries monitoring and real time alerting
			\item How they can be configured to monitor custom metrics
			\item How are alert thresholds defined
			\item How configuration and custom code is delivered to nodes (if requried)
			\item How the user configures the system
			\item How dependencies are handled
		\end{itemize}
	
	\subsection{Nagios}
	\paragraph{Timeseries monitoring and real time alerting}
	Nagios is primarily focused at real time alerting and therefore has very little in the way of 
	timeseries monitoring.  Additional plugins are available which can be used to graph metrics
	over time but these cannot be used to make decisions on the status of a given system or service.
	All that is supported in terms of alerting on historical data is to refrain from alerting until
	a given condition has been observed in the previous $n$ checks, there is no support for alerting
	based on trends in historical data.  Supports basic display of changes in state of hosts/services
	over time but not individual metrics.
	
	\paragraph{Support for custom metrics}
	Nagios has support for custom metrics through the NRPE (Nagios Remote Plugin Executor) plugin.
	These plugins can be any sort of executable which prints out a message to represent the data read
	as well as a specific exit code which defines the status, for example "OK", "Critical"
	.etc
	
	\paragraph{Alert threshold definition}
	Thresholds for NRPE agents must be set on the remote server itself.  These thresholds are passed
	into the remote plugin as an argument when it is executed and are used internally by the script to
	output the approprate alert level.
	
	\paragraph{Code/Config delivery to nodes}
	Nagios does not have any in built functionality to distribute configuration files or plugin code to
	remote nodes. In order to automate this, additional software such as Puppet would be required.
	
	\paragraph{How the user configures the system}
	Configuration for Nagios is primarily managed through text files stored on disk.  Third party
	configuraiton tools are available to allow the system to be configured through a web interface.
	Configuration lives on both the Nagios server as well as on the machines being monitored.
	
	\paragraph{How dependencies are handled}
	Rigid tree - No way to define that a service/host is dependant on a given host OR another host being
	available.  This reduces its usefulness in modern networks where redundancy and failover is
	commonplace. These are defined in config files that live on the Nagios server.
	
	\subsection{Icinga 2}
	\paragraph{Timeseries monitoring and real time alerting}
	Like Nagios, Icinga's primary focus is around real time alerting however it has now introuced support
	for graphing performance metrics.  No support for alerting based on trends in historical data beyond
	Nagios's idea of a state change changing to HARD once it has been observed $n$ times.
	
	\paragraph{Support for Custom Metrics}
	Custom metrics can be written as scripts in any language as long as they echo a message to describe
	the status of what they are checkign and use a certain exit code to define the status e.g. "OK",
	"CRITICAL", "WARNING".etc.
	
	\paragraph{Alert threshold definition}
	Thresholds are passed into the check commands as command line arguments.  Therefore they are defined
	on the machines being monitored individually.
	
	\paragraph{Code/Config delivery to nodes}
	Icinga 2 does not have any built-in mechanism to distribute code and config files to remote hosts.
	In order for this to be achieved, additional software such as Puppet would be required.
	
	\paragraph{How the user configures the system}
	Configuration is managed through configuration files stored on disk.  Configuration files exist both
	on the Icinga server as well as the nodes being monitored.
	
	\paragraph{How dependencies are handled}
	Same as Nagios with a rigid tree.  No way to define a host/service as being dependent on one of 
	several different machines as is common in modern environments with redundancy/failover systems.
	
	\subsection{Munin}
	\paragraph{Timeseries monitoring and real time alerting}
	Munin is targeted primarily as a timeseries monitoring tool.  It therefore has good functionality for
	graphing data over time.  However, it does not have much power in the way of alerting other than some
	very basic functionality where plugins must manually send out emails/syslog alerts.  This is not
	really sufficient for any sort of production use and it instead recommended to use a tool such as
	Nagios and push the data from Munin into it.
	
	\paragraph{Support for custom metrics}
	Munin has a simple interface for custom plugins where a plugin is a simple script that prints out the
	name and value of the data being collected.  This is then served by the Munin node which the Munin
	server contacts over the network to fetch values from the nodes.
	
	\paragraph{Alert threshold definition}
	Nothing built in, plugins however can create thresholds internally and use them for alerting (as
	detailed above) although this isn't really the intended use of Munin.
	
	\paragraph{Code/Config delivery to nodes}
	Nothing built in, code and config must be distributed manually or automatically through the use of
	additional software such as Puppet.
	
	\paragraph{How the user configures the system}
	Configuration is handled through text files stored on disk.  These are stored on both the Munin 
	server as well as machines being monitored.
	
	\paragraph{How dependencies are handled}
	No dependency functionality.
\end{document}
