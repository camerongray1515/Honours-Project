\documentclass[bsc,logo,twoside]{infthesis}

\title{Prophasis - An IT Infrastructure Monitoring Solution - Interim Report}
\author{Cameron Gray}
\abstract{Prophasis is an IT infrastructure monitoring system that is designed
to suit small to medium size businesses where a system needs to be intuitive to
manage. Management of the entire system can therefore be handled from a single,
responsive web interface. It is also suitable as a one-stop tool with support
for both time series monitoring in addition to real time alerting.
Traditionally two different tools would be needed to gain this level of
monitoring.}

\begin{document}
\maketitle
\tableofcontents

\chapter{Interim Report}
\section{Current Progress}
\subsection{Agent}
\paragraph*{}
	I have built a complete monitoring agent which runs on remote hosts being
	monitored. This exposes an HTTPS API which is used to interface with the
	agent.  The agent can receive plugin code over the network to allow the
	installation and update of custom plugins and can then execute them in order
	to collect data about the machine.  This data is then returned using the API.
	
\subsection{Core}
\paragraph*{}
	The core runs on the node doing the monitoring and communicates with the
	remote agents in order to trigger and collect data from checks.  It also
	handles scheduling to run checks at the appropriate times.

\paragraph*{}
	The core is
	multi threaded so it can quickly check multiple hosts at the same time, this
	prevents long running checks from slowing down the system as a whole.  A new
	thread is spawned for every machine that is currently being checked and then
	each thread will run each check on that machine sequentially. This means that
	there is no risk of multiple checks being performed on a single host at any
	one time, this prevents issues such as checks conflicting or overloading the
	machine being monitored.
	
\subsection{Web}
\paragraph*{}
	I have built a large portion of the web interface used to manage the system,
	this provides an easy to use interface for configuring the monitoring as well
	as dashboards that show the results of checks and to highlight any issues.
	The web interface was built to be responsive from the start and therefore
	works extremely well on mobile devices without any loss in functionality.
	
\subsection*{Plugins}
\paragraph*{}
	I have designed a structure and interface for the plugins that are used to
	actually perform the checks on the remote machines. I have then built several
	sample plugins to demonstrate their capability to monitor different aspects of
	a machine such as the CPU load, memory load, check if the machine will respond
	to a heartbeat signal and to check the status of a Linux MD raid device.
	
\paragraph*{}
	One challenge I found was how to build a system to classify the result of a
	plugin into whether it is "ok", "critical", "minor".etc.  My initial plan
	involved a web interface with various options however in order to make this
	reasonably powerful it started become extremely cluttered and complicated. I
	therefore designed a system where Lua code can be written to classify the
	data.  This is therefore very flexible and can be written at the same time
	as the rest of the plugin but can be customised on a per host basis (a VM
	host for example may have higher tolerances for memory usage compared to
	other types of machine).  The difficulty here was that I wanted this code to
	be editable from the web interface but if I were to use something such as
	Python this creates a large security risk due to being able to access core
	system APIs. I therefore settled on Lua as it is designed as an embedded
	language so therefore provides simple functionality for creating a sandbox
	execution environment with only explicitly assigned functions from being
	available in the code.
	
\subsection{Alerting}
\paragraph*{}
	I have designed a structure for and started work on implementing the web
	interface for managing alerts, i.e. deciding when to send out alerts related
	to changes within the monitoring system.

\section{Future Work}
\subsection{Finish Alerting}
\paragraph*{}
	Now that the web interface for managing alert conditions is complete I need to
	decide on how to support different methods of sending alerts, ideally through
	some sort of plugin system, this means I can not only use simple emails for
	alerts but I can also have scope for external services such as Twilio for SMS
	and telephone alerting and Pushbullet for push notifications.  One this design
	is decided on I can then implement it and therefore complete the alert
	functionality.

\subsection{Improve Dashboards}
\paragraph*{}
	There are currently a couple of simple dashboards for viewing the data
	collected but this needs to be improved to be able to present the data in
	various different ways such as being broken down by service or check. I will
	also investigate allowing the user to customise dashboards to their
	preferences.

\subsection{Distributed Monitoring}
\paragraph*{}
	Currently Prophasis (along with many other monitoring systems) only runs on a
	single host, this means that if this host or a network link between it and the
	machines being monitored fails, the monitoring will stop working. I would like
	to investigate the possibility of distributing this so there can be multiple
	machines performing the monitoring at the same time and then have the ability
	to easily recover (either automatically or manually) if a failure occurs
	somewhere in the monitoring system.
	
\paragraph*{}
	This could initially be accomplished by designating each monitoring node a set
	of machines that it is responsible for monitoring, this prevent conflicts
	caused by multiple nodes monitoring the same machine.  Data could then be kept
	in sync through using database level replication.  Care however would need to
	be taken to prevent conflicts between different database copies. The simplest
	way to do this would be to have only one machine that is allowed to update the
	configuration at any one time, all other monitoring nodes are simply allowed
	to insert the data they have collected and update data about the schedules for
	the hosts they are assigned to monitor.  This will ensure that no two machines
	will update the same database record at the same time.
	
\paragraph*{}
	Another tricky aspect of distributing the monitoring is how to handle failures
	of a monitoring node.  There are already well established algorithms that can
	be used to handle this which I would look into however an initial version of
	this could simply rely on the user manually clearing up after a failure where
	the machine that failed would no longer monitor the hosts it's assigned but
	the rest of them will continue to monitor their respective hosts as normal.
	
\subsection{Documentation and Packaging}
\paragraph*{}
	Due to the complex nature of this system I will need to write comprehensive
	documentation about setting up, managing and using the system.  This will
	range from how to simply use the web interface down to the level of technical
	documentation about how to write custom plugins.
	
\paragraph*{}
	I will also need to package the application up nicely in a distributable way
	with some sort of installer/setup script.  Currently I work by simply manually
	executing the Python source files inside a virtual environment which is ideal
	for development work but is not well suited for production use.  I will also
	need to investigate swapping out the Flask development webserver that
	currently powers the web interface for something more production ready.
	

\chapter{Introduction}
\section{Background}
\paragraph*{}
	% TODO: Reword this?
	In recent years, almost all businesses have been expanding their IT 
	infrastructure to handle the modern demand for IT systems.  As these systems
	grow and become increasingly important for business operation it is crucial
	that they are sufficiently monitored to prevent faults and periods of downtime
	going unnoticed.  There is already a large market of tools for monitoring
	IT systems however they are designed for use on massive scale networks managed
	by teams of specialised systems administrators.  They are therefore
	complicated to set up and manage and multiple tools are often required to gain
	a suitable level of monitoring.
	
\paragraph*{}
	For example, tools generally either fall into
	the category of real time alerting (i.e. telling someone when something
	breaks) and time series monitoring (i.e. capturing data about the performance
	of systems and presenting graphs and statistics based on it), there is a large
	gap in the market for tools that provide both of these in one package. This
	reduces the time required to manage the system as it eliminates the need to
	set up and configure two completely separate tools.

\paragraph*{}	
	These tools are also generally managed and configured through various
	configuration files split across different machines on the network. This means
	that in order to efficiently use these tools a configuration management system
	such as	Puppet must be used. In a small business with limited IT resources, a
	completely self contained system is often preferable.

\section{Improvements}
% TODO: Find a better place for this?
\paragraph*{}
	Prophasis is designed for use in a small to medium business with limited IT
	resources.  They may have a small IT team with limited resources or may not
	even have a dedicated IT team at all, instead relying on one or two employees
	in other roles who manage the business's IT systems on the side of their
	regular jobs. Therefore the system needs to be quick to deploy and manage with
	a shallow learning curve. In order to use the system efficiently there should
	be no requirement for additional tooling to be deployed across the company.
	
\subsection{Configuration Management}
\paragraph*{}
	It should be possible to manage the configuration of the system from a single
	location.  Prophasis therefore provides a responsive web interface where every
	aspect of the system's operation can be configured, Prophasis then handles
	distributing this configuration to all other machines in the system in the
	background. Custom code for plugins is handled in the same way; it is uploaded
	to the single management machine and is then automatically distributed to the
	appropriate remote machines when it is required.
	
\subsection{Time Series Monitoring \& Real Time Alerting}
\paragraph*{}
	Prophasis provides both the ability to alert administrators in real time when
	a fault is discovered with the system alongside functionality to collect
	performance metrics over time and use this data to generate statistics about
	how the system has been performing.  This time series data can be used to both
	investigate the cause of a failure in post-mortem investigations in addition
	to being able to be used to predict future failures by looking at trends in
	the collected data.
	
\subsection{Expandability}
\paragraph*{}
	It is important that a monitoring tool can be expanded to support the
	monitoring of custom hardware and software.  An example of this would be
	hardware RAID cards.  Getting the drive health from these types of devices
	can range from probing for SMART data all the way to communicating with the
	card over a serial port.  It is therefore crucial that Prophasis can be
	easily expanded to support custom functionality such as this. Therefore 
	Prophasis supports a system of custom "plugins" which can be written and
	uploaded to the monitoring server where they can then be configured to monitor
	machines. These plugins are designed to be self contained and to follow a well
	defined and documented structure.  This provides scope for a plugin
	"marketplace" much like there already exists for plugins for software such as
	Wordpress and Drupal allowing plugins to be easily shared and installed for
	monitoring various pieces of hardware and software, therefore eliminating the
	need for every user to implement custom monitoring code for the systems they
	are using.
	
\chapter{Design}
\section{Technology Choice}
\subsection{Why Python?}
\subsection{Why HTTPS?}

\section{System Structure}
\begin{itemize}
	\item Explain different components
	\item Diagram
	\item Why separate?
\end{itemize}

\section{Monitoring Methodology}
\subsection{Host Management}
\begin{itemize}
	\item Stuff about hosts and host groups
\end{itemize}

\subsection{Plugins}
\begin{itemize}
	\item What is a plugin?
\end{itemize}

\subsection{Checks}
\begin{itemize}
	\item What is a check?
\end{itemize}

\subsection{Schedules}
\begin{itemize}
	\item What is a schedule?
\end{itemize}

\subsection{Services}
\begin{itemize}
	\item What is a service?
	\item Dependencies \& Redundancy Groups
	\item Why use dependencies?
	\begin{itemize}
		\item Alert only if service functionality is severely impacted
		\item Prevents unnecessary alerts due to failures in redundant infrastructure
		\item Provides clearer view of impact of a given failure
	\end{itemize}
\end{itemize}

\subsection{Alerts}
\begin{itemize}
	\item What is an alert?
	\item Separate alerts for different checks/hosts/plugins/services
	\item State transition restrictions - Reduce unnecessary alerts
\end{itemize}

% TODO: Is this more like implementation?
\section{Plugin Interface}
\begin{itemize}
	\item Explain structure of a plugin
	\item UML Diagram?
	\item Why Lua for classification logic?
\end{itemize}
\end{document}
